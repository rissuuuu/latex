\begin{onehalfspacing}
\chapter{Introduction}
\section{Introduction to Operating System} 
Operating System serves as an interface between user and hardware because applications software package does not communicate directly with the hardware. Operating System plays vital role in 
managing computer resources like memory and storage. Operating system is responsible for managing all the components of hardware and mainly handling processes. It contains a supervisor program that remains in memory and controls the entire operating System and loads into memory from disk storage as per the demand by CPU. Operating system communicates with processor which is responsible for executing processes whereas memory is responsible for loading a process into CPU for execution. Operating system handles both application softwares as well as system softwares.\newline

\par Some of the functions of Operating System are listed below:
\begin{itemize}
    \item \text{Processor Management}
    \item \text{Memory Management}
    \item \text{Device Management}
    \item \text{File Management}
\end{itemize} 

\subsection{Operating System Simulation}
Simulation is the process of modeling a real phenomenon of a system with some set of computer programs. Simulation allows user to observe working behavior of complex systems with the help of simple software. Operating system simulation varies with aspects like CPU and processes, Physical and virtual memory along with paging, File systems, etc. Process simulation includes simulating behavior of CPU with the help of scheduling algorithms and process executions. Memory simulation includes simulation of Physical and virtual memory with the help of various page replacement algorithms.

\par By implementing any of these simulation concept, its very useful to understand individual aspect of computer system and its working mechanism.


\section{Motivation}


\section{Problem Statement}


\section{Thesis Organization}

\end{onehalfspacing}