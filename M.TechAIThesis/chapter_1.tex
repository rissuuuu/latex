\begin{onehalfspacing}
\chapter{Introduction}
\section{Introduction to Operating System} 
Operating System serves as an interface between user and hardware because applications software package does not communicate directly with the hardware. Operating System plays vital role in 
managing computer resources like memory and storage. Operating system is responsible for managing all the components of hardware and mainly handling processes. It contains a supervisor program that remains in memory and controls the entire operating System and loads into memory from disk storage as per the demand by CPU. Operating system communicates with processor which is responsible for executing processes whereas memory is responsible for loading a process into CPU for execution. Operating system handles both application softwares as well as system softwares.\newline

\par Some of the functions of Operating System are listed below:
\begin{itemize}
    \item \text{Processor Management}
    \item \text{Memory Management}
    \item \text{Device Management}
    \item \text{File Management}
\end{itemize} 

\subsection{Operating System Simulation}
Simulation is the process of modeling a real phenomenon of a system with some set of computer programs. Simulation allows user to observe working behavior of complex systems with the help of simple software. Operating system simulation varies with aspects like CPU and processes, Physical and virtual memory along with paging, File systems, etc. Process simulation includes simulating behavior of CPU with the help of scheduling algorithms and process executions. Memory simulation includes simulation of Physical and virtual memory with the help of various page replacement algorithms.
\par Simulation solves real-world problems safely and efficiently and if it is compared to real world computer systems, its very difficult to understand how computations are happening and how processes are being executed. In a real world, when a big processes need to be executed, and if it cannot not be loaded into main memory, the concept of paging comes into existence where a process is divided into number of pages. So virtual memory simulation provides valuable solution by giving clear insights into complex memory operations. Particular memory simulation enables experimentation on a valid representation of physical memory and virtual memory. Virtual memory is used to emulate the computer's Random Access Memory (RAM)

\par By implementing any of these simulation concept, its very useful to understand individual aspect of computer system and its working mechanism.

\section{Motivation}
In order to know how a computer system works in a visual representational architecture, we have chosen to develop a simple simulator that describes the process, scheduling and memory management concept in Operating System. Teaching Operating system is a hectic work and it is also difficult to show the flow of execution in theoretical way. Due to less available resource, it is also not practically possible to represent whole working mechanism of an Operating System with one or two practical sessions. Taking this into consideration, here comes a simulator software handy during teaching and learning process as it describes the flow of execution in a good understandable form. There are some of Operating System (OS) simulation software used for teaching but each of them are focusing in specific part of OS and also implemented in their own way where it is found that process scheduling simulator only focuses in process management and less focuses in other part and vice versa. Upon available teaching tools for Operating Systems, we have chosen to create our own tool which implements process creation, process execution, process scheduling, memory management and gives us comparison between various page replacement algorithms.

\section{Problem Statement}
The main purpose of this project aims to implement some of the concepts of Operating System like process creation and scheduling, locality of reference, memory management and page replacements using python programming language. Here we try to create simple simulation tool covering above mentioned topics.

\section{Thesis Organization}
The dissertation is as follows, Chapter 2 deals with literature survey where study on various research papers is done in various Operating system teaching tools proposed by researchers from around the globe. Chapter 3 deals with some of the methodologies and its implementation in detail about proposed model. Chapter 4 deals with results and comparison of performance among the chosen algorithms. Chapter 5 deals with future work and conclusion.

\end{onehalfspacing}    